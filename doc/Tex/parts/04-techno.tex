\chapter{Технологический раздел}

В данном разделе будут выбраны средства реализации, которые будут использованы при разработке программного обеспечения.
Также будут продемонстрированы интерфейс и работа программы.

\section{Выбор средств реализации}

Далее будет обоснован выбор языка программирования, графического интерфейса для работы в трехмерном пространстве и среды разработки.

\subsection{Выбор графического интерфейса для работы в трехмерном пространстве} 

В качестве графического интерфейса для работы в трехмерном пространстве был выбран OpenGL в силу следующих причин \cite{OpenGL}:
\begin{enumerate}
	\item рендеринг на графическом процессоре (GPU) осуществляется быстрее, чем на центральном процессоре (CPU) \cite{gpuFaster};
	\item большое сообщество разработчиков, а также множество материалов в открытом доступе.
\end{enumerate}

\subsection{Выбор языка программирования} 

В качестве языка программирования для разработки программного обеспечения был выбран C++ в силу следующих причин:
\begin{enumerate}
	\item язык C++ изучался в рамках курса <<Объектно-Ориентированное Программирование>>;
	\item язык C++ обладает высокой производительностью;
	\item наличие библиотек для удобной работы с OpenGL и создания оконных приложений: glad, GLFW, GLEW \cite{glLibs}.
\end{enumerate}

\clearpage

\subsection{Выбор среды разработки} 

В качестве среды для разработки программного обеспечения была выбрана Visual Studio 2022 в силу следующих причин \cite{vs2022}:
\begin{enumerate}
	\item поддержка отладки графических приложений, а также большое количество инструментов для поиска и анализа ошибок;
	\item удобная интеграция с языком C++;
	\item наличие расширений для гибкой интеграции с OpenGL.
\end{enumerate}

\section{Описание интерфейса пользователя}

При запуске разработанного ПО, на экране появляется вспомогательное окно <<Главное меню>> для взаимодействия с программой.
Меню содержит следующие разделы, представленные на рисунке \ref{img:allmenu.png}:
\begin{itemize}
	\item <<Препятствия>>~--- раздел для управления препятствиями сцены;
	\item <<Освещение>>~--- раздел для управления освещением;
	\item <<Источник звука>>~--- раздел для управления источниками звуковых волн.
\end{itemize}
\includeimage
	{allmenu.png}
	{f}
	{h}
	{0.6\textwidth}
	{Окно <<Главное меню>> для взаимодействия с программой}

\clearpage

Раздел <<Препятствия>>, представленный на рисунке \ref{img:obstaclesmenu.png}, предоставляет пользователю возможность создавать препятствия, указывая при этом их параметры: цвет, позицию, масштаб и угол поворота.
Также предусмотрена возможность удаления отдельных препятствий.
\includeimage
	{obstaclesmenu.png}
	{f}
	{h}
	{0.6\textwidth}
	{Раздел <<Препятствия>>}
	
\clearpage
	
Раздел <<Освещение>>, представленный на рисунке \ref{img:lightingmenu.png}, предоставляет пользователю возможность задавать положение источника света и его цвет.
Также предусмотрена возможность удаления источника света.
\includeimage
	{lightingmenu.png}
	{f}
	{h}
	{0.6\textwidth}
	{Раздел <<Освещение>>}
	
\clearpage

Раздел <<Источник звука>>, представленный на рисунке \ref{img:wavesourcemenu.png}, предоставляет пользователю возможность устанавливать источники звуковых волн и задавать скорость распространения звука из этих источников, после чего запускать процесс визуализации путем нажатия кнопки <<Испустить волну из источников звука>>.
Также предусмотрена возможность удаления отдельных источников звуковых волн.
	\includeimage
	{wavesourcemenu.png}
	{f}
	{h}
	{0.6\textwidth}
	{Раздел <<Источник звука>>}

\section{Демонстрация работы программы}

На рисунке \ref{img:demoonlywaves.png} продемонстрирована работа программы при испускании нескольких звуковых волн из источников звука.
\includeimage
	{demoonlywaves.png}
	{f}
	{h}
	{1\textwidth}
	{Демонстрация работы программы при испускании нескольких звуковых волн из разных источников}
	
\clearpage

На рисунке \ref{img:demowithobst.png} продемонстрирована работа программы при испускании нескольких звуковых волн из источников звука, которые отражаются от установленного препятствия.
\includeimage
	{demowithobst.png}
	{f}
	{h}
	{1\textwidth}
	{Демонстрация работы программы при испускании нескольких звуковых волн из разных источников}
	
\clearpage
	
\section{Диаграмма классов}

На рисунке \ref{img:classes.pdf} представлена диаграмма классов разработанного программного обеспечения.
\includeimage
	{classes.pdf}
	{f}
	{h}
	{1\textwidth}
	{Диаграмма классов разработанного программного обеспечения}

\section{Вывод}

В данном разделе были выбраны средства для разработки программного обеспечения:
\begin{enumerate}
	\item графический интерфейс для работы в трехмерном пространстве;
	\item язык программирования;
	\item среда разработки.
\end{enumerate}
Также был описан интерфейс пользователя, приведены примеры работы программы и представлена диаграмма классов.