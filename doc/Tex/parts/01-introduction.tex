\chapter*{ВВЕДЕНИЕ}
\addcontentsline{toc}{chapter}{ВВЕДЕНИЕ}

Компьютерная графика играет важную роль в современной технологической среде, находя применение в различных сферах, от разработки инженерного проектирования до компьютерных анимаций в мультфильмах.
В наше время цифровое графическое представление достигло максимальной реалистичности, что требует огромных мощностей \cite{usage}.

Целью данной курсовой работы является разработка ПО, которое позволит визуализировать процесс распространения звуковых волн в замкнутом пространстве.

Для достижения поставленной цели необходимо решить следующие задачи:
\begin{enumerate}
	\item проанализировать предметную область;
	\item формализовать объекты сцены;
	\item рассмотреть известные методы описания моделей на сцене, подходы и алгоритмы удаления невидимых линий и поверхностей, модели освещения, методы закраски;
	\item спроектировать программное обеспечение для визуализации распространения звуковой волны;
	\item выбрать средства реализации;
	\item исследовать характеристики разработанного программного обеспечения.
\end{enumerate}