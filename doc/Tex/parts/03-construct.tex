\chapter{Конструкторский раздел}

В данном разделе представлены требования к программному обеспечению и формальное описание алгоритмов, выбранных в аналитическом разделе.

\section{Требования к программному обеспечению}

Разработанная программа должна обладать следующим функционалом:
\begin{enumerate}
	\item перемещение, поворот и масштабирование сцены;
	\item добавление и удаление препятствий на сцене;
	\item добавление и удаление точечных источников звуковых волн;
	\item добавление и удаление единственного точечного источника освещения.
\end{enumerate}

\section{Общий алгоритм построения изображения}

Общий алгоритм решения поставленных задач представлен на рисунке \ref{img:all.pdf}. 
На вход подаются точки моделей (препятствий), камеры, волны (если она есть).
На выходе получается изображение в текущий момент времени.
\includeimage
	{all.pdf}
	{f}
	{h}
	{1\textwidth}
	{Схема общего алгоритма построения изображения}

\clearpage
	
\section{Алгоритм изображения звуковой волны}

Схема алгоритма изображения звуковой волны представлена на рисунке \ref{img:sound.pdf}.
На вход подаются точки, образующие волну, и их скорости.
На выходе получается сцена с изображенной звуковой волной.
\includeimage
	{sound.pdf}
	{f}
	{h}
	{0.6\textwidth}
	{Схема алгоритма изображения звуковой волны}
	
\clearpage
	
\section{Алгоритм простой закраски}

Схема алгоритма простой закраски представлена на рисунке \ref{img:simple.pdf}.
На вход подаются точки модели (препятствия) и ее цвет.
На выходе получается сцена с закрашенной моделью.
\includeimage
	{simple.pdf}
	{f}
	{h}
	{0.5\textwidth}
	{Схема алгоритма простой закраски}
	
\section{Алгоритм, использующий Z-буфер}

Схема алгоритма, использующего Z-буфер, представлена на рисунке \ref{img:zbuf.pdf}.
На вход подаются точки модели сцены, буфер кадра.
На выходе получается сцена без невидимых линий \cite{baseInvisible}.
\includeimage
	{zbuf.pdf}
	{f}
	{h}
	{0.7\textwidth}
	{Схема алгоритма, использующего Z-буфер}
	
\clearpage

\section{Алгоритм вычисления освещенности по модели Фонга}

Схема алгоритма вычисления освещенности по модели Фонга представлена на рисунке \ref{img:phong.pdf}.
На вход подаются точки модели сцены и ее цвет, положение источника света.
На выходе получается освещенная модель \cite{baseLight}.
\includeimage
	{phong.pdf}
	{f}
	{h}
	{0.35\textwidth}
	{Схема алгоритма вычисления освещенности по модели Фонга}
	
\section{Выбор типов и структур данных}

Для решения поставленных задач курсового проекта необходимо определить представление объектов разрабатываемой программы.
В таблице \ref{table:types} представлены объекты и выбранные для них типы и структуры данных.
\begin{table}[h!]
	\begin{center}
		\caption{\label{table:types} Таблица объектов и выбранных типов и структур данных}
		\begin{tabular}{|p{230pt}|p{230pt}|}
			\hline
			Объект & Типы и структуры данных \\ \hline
			Точка в трехмерном пространстве & Вектор из трех элементов: $x$, $y$, $z$ \\ \hline
			Скорость точки & Вектор из трех элементов: $sx$, $sy$, $sz$ \\ \hline
			Вершина & Точка в трехмерном пространстве \\ \hline
			Нормаль & Вектор из трех элементов: $nx$, $ny$, $nz$ \\ \hline
			Направление & Вектор из трех элементов: $dx$, $dy$, $dz$ \\ \hline
			Грань & Структура данных, содержащая: массив из 6 вершин и нормаль \\ \hline
			Звуковая волна & Структура данных, содержащая: массив граней и массив скоростей точек \\ \hline
			Препятствие & Массив граней \\ \hline
			Замкнутое пространство (комната) & Массив граней \\ \hline
			Источник звука & Точка в трехмерном пространстве \\ \hline
			Камера & Структура данных, содержащая: направление и точку в трехмерном пространстве \\ \hline
		\end{tabular}
	\end{center}
\end{table}

\section{Вывод}

В данном разделе были сформированы требования к программному обеспечению.
Также были выбраны типы и структуры данных и описаны следующие алгоритмы:
\begin{itemize}
	\item общий алгоритм построения сцены;
	\item алгоритм изображения звуковой волны;
	\item алгоритм простой закраски;
	\item алгоритм, использующий Z-буфер;
	\item алгоритм вычисления освещенности по модели Фонга.
\end{itemize}