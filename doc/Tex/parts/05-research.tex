\chapter{Исследовательский раздел}

\section{Технические характеристики}

Технические характеристики устройства, на котором проводилось исследование:
\begin{itemize}
	\item центральный процессор: Intel Core i5-10300H CPU @ 2.50 ГГц;
	\item графический процессор: Nvidia GeForce GTX 1650;
	\item оперативная память: 16 Гб;
	\item операционная система: Ubuntu 22.04.3 LTS.
\end{itemize}

Во время проведения измерений времени ноутбук был подключен к сети электропитания и был нагружен только системными приложениями.

\section{Проведение исследования}

Целью исследования является определение зависимости времени генерации кадра сцены от количества препятствий.
Все расставляемые препятствия имеют одинаковую форму и цвет.
Для каждого количества объектов препятствия расставляются на сцене 20 различными способами, после чего берется среднее время построения.
Количество испускаемых волн постоянно.

Результаты зависимости времени генерации кадра сцены от количества препятствий представлены в таблице \ref{table:measures} и на рисунке \ref{img:measures.pdf}.
\begin{table}[h!]
	\begin{center}
		\caption{\label{table:measures} Количественные данные, полученные в результате исследования}
		\begin{tabular}{|p{225pt}|p{225pt}|}
			\hline
			Количество препятствий (кол.) & Время генерации кадра (мкс) \\ \hline
			1 & 3.241 \\ \hline
			2 & 6.112 \\ \hline
			3 & 9.445 \\ \hline
			4 & 13.567 \\ \hline
			5 & 17.131 \\ \hline
			6 & 21.522 \\ \hline
			7 & 25.525 \\ \hline
			8 & 29.332 \\ \hline
			9 & 34.124 \\ \hline
			10 & 37.746 \\ \hline
		\end{tabular}
	\end{center}
\end{table}

\includeimage
	{measures.pdf}
	{f}
	{h}
	{1\textwidth}
	{Зависимость времени генерации кадра сцены от количества препятствий}
	
\clearpage
	
\section{Вывод}

В данном разделе были описаны технические характеристики устройства, на котором проводилось исследование.

Исходя из полученных в таблице \ref{table:measures} результатов, был сделан вывод, что время генерации кадра сцены увеличивается линейно с увеличением количества препятствий на сцене.